\documentclass[a4paper,10pt]{article}
\usepackage[utf8]{inputenc}
\setlength{\parindent}{0pt}
\usepackage{soul}
\usepackage{xcolor}

\setstcolor{red}

% Title Page
\title{Response to referee reports \\ AP11756/Montanari }
\author{C.C. Montanari, A. M. P. Mendez, D. M. Mitnik, J. E. Miraglia, \\
P.A. Miranda, R. Correa, J. Wachter, M. Aguilera, \\
E. Alves, N. Catarino and R.C. da Silva}
\date{\today}

\begin{document}
\maketitle

% ----------------------------------------------------------------------
\section{Reply to report of the First Referee}
% ----------------------------------------------------------------------

\textsl{1. Since only protons are considered as projectiles, the title 
needs to be modified accordingly.}

\vspace{0.1cm}
The title has been modified following the referee's recommendation to
``Stopping power of hafnium by proton impact and the importance of 
relativistic $4f$ electrons''.

% ----------------------------------------------------------------------
\vspace{0.25cm}
\textsl{2. It appears that calculations assume bare protons. According 
to CasP, neutrals play a considerable role in charge equilibrium around 
and below the Bragg peak. Some explanation is needed here.}

\vspace{0.1cm}
{\color{red}Reply here.}

% ----------------------------------------------------------------------
\vspace{0.25cm}
\textsl{3. Thickness measurements by Rutherford backscattering rely 
‘heavily’ on the scattering cross section, while energy loss is a minor 
correction depending on foil thickness (p.1, 2nd paragraph).}

\vspace{0.1cm}
{\color{red}Reply here.}

% ----------------------------------------------------------------------
\vspace{0.25cm}
\textsl{4. In order to get a more precise thickness value of the Hf 
foil, the authors use the energy loss of 5.486 MeV alphas as a standard, 
with SRIM as a reference. SRIM is based on empirical data, and for alpha 
particles in Hf, the IAEA database lists one single data point for He 
in Hf. Considering the spread in published stopping cross sections for 
e.g. He in Au, listed in the IAEA database, some explanation is needed 
to justify an error estimate of only 5\% (1st paragraph in section 
IIB).}

\vspace{0.1cm}
{\color{red}Reply here.}

% ----------------------------------------------------------------------
\vspace{0.25cm}
\textsl{5. In connection with the discussion of relativity and 
outer-shell electrons (p.3) it might be instructive to mention that 
this is the reason for the existence of relativistic quantum chemistry 
programs. Since the authors find it surprising that the effect increases 
in magnitude from inner to outer shells, the reader needs an 
explanation. I should think the effect of deviations in the screening 
add up shell by shell, but I may be wrong. Moreover, the sign of the 
discrepancy in figure 2a deserves attention, which seems to change from 
positive up to 4p- to negative from 5p+.}

\vspace{0.1cm}
The referee is correct. We included the following modifications in the text
(lines 200-214):

\vspace{0.1cm}
{\small ``One would expect the relativistic contributions to only affect 
significantly the inner shells. The relativistic binding energies of the 
electrons belonging to these shells are 2.5\% from the experimental values,
while the non-relativistic ones are within $\sim$10\%. Moreover, we notice 
the relativistic corrections greatly influences the outer $5p$ and $4f$ 
shells. This effect may be a consequence of the orthogonality with the 
relativistic inner shells. Also, the sign of the binding energy deviations 
is inverted near the FEG limit, i.e. the electrons go from being more 
bounded --than the experimental values-- to being less bounded. 
This change of sign can be attributed to the fact that the experimental 
values correspond to hafnium in solid-state, while our theoretical 
calculations correspond to the element in the gas phase.''}

% ----------------------------------------------------------------------
\vspace{0.25cm}
\textsl{6. The importance of 4f-5p screening is documented clearly 
(p.3), but the reader will ask why this does not affect other shells. 
I assume that the effect is important when two or more subshells have 
similar energies and, hence, overlap in real and velocity space. If so, 
I suggest to mention this. If not, another explanation is needed.}

\vspace{0.1cm}
{\color{red}Reply here.}

% ----------------------------------------------------------------------
\vspace{0.25cm}
\textsl{7. In view of the large contribution of the outermost shell, I 
suggest to include the core contribution in figure 3.}

\vspace{0.1cm}
{\color{red}Reply here.}

% ----------------------------------------------------------------------
\vspace{0.25cm}
\textsl{8. In addition to comparisons with experiments and empirical 
models, it would be instructive to see comparisons with theoretical 
models like CasP and DPASS in figure 4.}

\vspace{0.1cm}
{\color{red}Reply here.}




\newpage
% ----------------------------------------------------------------------
\section{Reply to report of the Second Referee}
% ----------------------------------------------------------------------

\textsl{1. Can the authors discuss how possible impurities in the sample and -especially- on both sides of the target would affect the final
stopping data? The authors must discuss it in the manuscript.}

\vspace{0.1cm}
{\color{red}Reply here.}

% ----------------------------------------------------------------------
\vspace{0.25cm}
\textsl{2. In the same context, the authors have to clarify whether potential
surface roughness and/or target non-uniformity might affect the final
results obtained in transmission method.}

\vspace{0.1cm}
{\color{red}Reply here.}

% ----------------------------------------------------------------------
\vspace{0.25cm}
\textsl{3. What is accuracy of the primary beam energy of the accelerator?
Moreover: when the target is placed and removed from the front of the
detector, does the detector itself need to unbiased and the chamber
opened? If so, might there be any offset ($\sim$keV) in the detection
system that would affect the energy distribution peaks, hence the
Gaussian position from the fits (Fig. 1)?}

\vspace{0.1cm}
{\color{red}Reply here.}

% ----------------------------------------------------------------------
\vspace{0.25cm}
\textsl{4. The experimental data above $\sim$1.5 MeV is systematically 
below the data from Ref. [7], hence SRIM semi-empirical approach, and 
also below the authors’ theory output. At these energies (and higher), 
the agreement between experiment and theory is expected to be better than
a few percent (that is why you used stopping power for 5.486 MeV alpha
on Hf to obtain the target thickness). Can the authors explain why
there is such systematic difference for the present measurements using
protons?}

\vspace{0.1cm}
{\color{red}Reply here.}

% ----------------------------------------------------------------------
\vspace{0.25cm}
\textsl{5. The authors claim an overall uncertainty of 5\%, but it is not clear
how they obtained this value, or whether only statistical
contributions were taken into account. Nevertheless, uncertainties of
measurements have to be treated following the Guide to the Expression
of Uncertainty in Measurement, International Organization for
Standardization, Geneva Switzerland, 1995. Especially for stopping
power measurements in transmission approach, there are other sources
of uncertainties that might potentially affect the final results
systematically (see comments 1, 2 and 3), and they must be discussed
in more details.}

\vspace{0.1cm}
{\color{red}Reply here.}

% ----------------------------------------------------------------------
\vspace{0.25cm}
\textsl{6. Can the present theory in the manuscript be compared to other
models (e.g., CasP and DPASS), especially towards low energies? Aiming
thus a constructive discussion on the overall-agreement among
different models to predict SCS of “challenge” materials, as
transition metals (f-subshell)?}

\vspace{0.1cm}
{\color{red}Reply here.}

% ----------------------------------------------------------------------
\vspace{0.25cm}
\textsl{7. In the last sentence of Sec. IV, the authors point out for a need
of more experimental data below 100 keV H+ but, in my opinion, more
data are needed even at higher energies, e.g., nearby (and slightly
above) the Bragg peak... Please enhance discussions about it.}

\vspace{0.1cm}
{\color{red}Reply here.}




\newpage
% ----------------------------------------------------------------------
\section{Reply to report of the Third Referee}
% ----------------------------------------------------------------------

\textsl{
- Regarding the theoretical model that describes the contribution to
the stopping power of the bound electrons, the authors emphasize the
importance of including relativistic effects in the description of the
binding energies and densities of the Hf 1s-4f states. However, it is
not clear if the calculations performed by the authors are done for
gas phase Hf atoms or for the solid Hf as in the experiments of Ref.
[38]. In case of being for the former, the slightly worse agreement
found for the 4f states as compared to the experimental values could
be related to this point.}

\vspace{0.1cm}
Our calculations are performed for gas phase Hf atoms. We have clarify
this in the manuscript (lines 211-214).

% ----------------------------------------------------------------------
\vspace{0.25cm}
\textsl{
1. The results from the three different theoretical calculations shown
in Fig. 3 are not completely clear. The authors emphasize the
importance of considering both screening effects and relativistic
effects for the 4f and 5p electrons, but it is not clear if later in
the text the authors use indistinctly the terminology ``relativistic''
and ``screening'' to refer to the same issue. For instance, are the ML
results shown in Fig. 3 calculated with the relativistic binding
energies and densities, but with and without including screening
corrections, or do the two ML curves differ in whether the
relativistic corrections are or not included? It is neither clear what
description of the bound electrons is used in the curve labeled as
``SPCC(FEG)+SLPA(bound)''.}

\vspace{0.1cm}
{\color{red}Reply here.}

% ----------------------------------------------------------------------
\vspace{0.25cm}
\textsl{
2. If all the results of Fig. 3 were calculated for the relativistic
description, it would be also interesting to show the results obtained
with the non-relativistic description in order to demonstrate the
importance of this correction.}

\vspace{0.1cm}
{\color{red}Te mando adjunto las funciones de onda y energías de ligaduras
del caso no relativista (NR). Habría que ver como queda.}

% ----------------------------------------------------------------------
\vspace{0.25cm}
\textsl{
3. Furthermore, it would also be meaningful to show the results from
the ML(FEG) calculation alone in order to identify better the
contribution of the bound electrons to the SP.}

\vspace{0.1cm}
{\color{red}Reply here.}

% ----------------------------------------------------------------------
\vspace{0.25cm}
\textsl{
4. When comparing the new and existing experimental data, clarify what
experimental technique was used in Ref. 7 to measure the stopping
power.}

\vspace{0.1cm}
{\color{red}Reply here.}

% ----------------------------------------------------------------------
\vspace{0.25cm}
\textsl{
5. The data by Sirotonin et al. plotted in Fig. 4 seem to suggest that
the stopping power maximum would be around 100-200 keV, while the
maximum predicted in the present work by the theoretical calculations
is located at slightly smaller energies. Please comment on this point
and discuss possible reasons for this discrepancy: What are the main
limitations in the theoretical model that can be causing this energy
shift?}

\vspace{0.1cm}
{\color{red}Reply here.}

% ----------------------------------------------------------------------
\vspace{0.25cm}
\textsl{
6. When reviewing on the experimental work performed on Hf oxides in
the introduction, note that there is a quite recent work by Roth and
co-workers published in Phys. Rev. Lett. 119, 163401 (2017), in which
the SP of protons is precisely measured in this target.}

\vspace{0.1cm}
We thank the referee for suggesting the missing reference. We have now
include it in the revised manuscript.

% ----------------------------------------------------------------------
\vspace{0.25cm}
\textsl{Finally, there are few misprints in the generated pdf file:}

\textsl{- Y-axis labels in Figs. 3 an 4 are unreadable }
{\color{red} ¿Cómo quedó el preprint? El pdf que genera mi latex tiene 
bien los labels, así que debe un problema del generador de latex del PRA...}

\st{\textsl{- Page 3,2nd column, 2nd full paragraph: "Too assess"}}

\st{\textsl{- Caption of Fig.2: "hollow circles" -$>$ "open circles" or
"unfilled circles"}}


\end{document}          
