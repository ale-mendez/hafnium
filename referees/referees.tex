
----------------------------------------------------------------------
Report of the First Referee -- AP11756/Montanari
----------------------------------------------------------------------

The stopping power of hafnium and the importance of relativistic 4f electrons
Montanari et al, Phys. Rev. A

This manuscript continues previous theoretical work by members of this 
group on stopping of protons in heavy metals, including measurements in 
hafnium over an energy range ∼ 0.5 − 2MeV. This work is of interest not 
only in view of a notorious lack of experimental data in that part of 
the periodic table, but also because of an interesting shell structure, 
the significance of which is a central part of this study. While 
publication in PRA is indicated, I have a number of requests, questions 
and suggestions, which need consideration.

1. Since only protons are considered as projectiles, the title needs to 
be modified accordingly.

2. It appears that calculations assume bare protons. According to CasP, 
neutrals play a considerable role in charge equilibrium around and 
below the Bragg peak. Some explanation is needed here.

3. Thickness measurements by Rutherford backscattering rely ‘heavily’ 
on the scattering cross section, while energy loss is a minor correction 
depending on foil thickness (p.1, 2nd paragraph).

4. In order to get a more precise thickness value of the Hf foil, the 
authors use the energy loss of 5.486 MeV alphas as a standard, with 
SRIM as a reference. SRIM is based on empirical data, and for alpha 
particles in Hf, the IAEA database lists one single data point for He 
in Hf. Considering the spread in published stopping cross sections for 
e.g. He in Au, listed in the IAEA database, some explanation is needed 
to justify an error estimate of only 5\% (1st paragraph in section IIB).

5. In connection with the discussion of relativity and outer-shell 
electrons (p.3) it might be instructive to mention that this is the 
reason for the existence of relativistic quantum chemistry programs. 
Since the authors find it surprising that the effect increases in 
magnitude from inner to outer shells, the reader needs an explanation. 
I should think the effect of deviations in the screening add up shell 
by shell, but I may be wrong. Moreover, the sign of the discrepancy in 
figure 2a deserves attention, which seems to change from positive up to 
4p- to negative from 5p+.

6. The importance of 4f-5p screening is documented clearly (p.3), but 
the reader will ask why this does not affect other shells. I assume 
that the effect is important when two or more subshells have similar 
energies and, hence, overlap in real and velocity space. If so, I 
suggest to mention this. If not, another explanation is needed.

7. In view of the large contribution of the outermost shell, I suggest 
to include the core contribution in figure 3.

8. In addition to comparisons with experiments and empirical models, 
it would be instructive to see comparisons with theoretical models like 
CasP and DPASS in figure 4.

----------------------------------------------------------------------
Report of the Second Referee -- AP11756/Montanari
----------------------------------------------------------------------

C. C. Montanari and co-authors report experimental and theoretical
stopping cross section data of protons in hafnium. Experimental data
are obtained from transmission experiments in the energy interval of
[600 – 2500] keV, whereas theoretical results are based on a
combination of different approaches: shell-wise local plasma
approximation model (SLPA - from the same author) for higher impact
ion velocities (around and above Bragg peak) and two free electron gas
models (FEG) for low impact velocities (below the Bragg peak). The
study is thorough and the manuscript well organized, emphasizing on
the importance of the material under analysis. The experimental and
theoretical approaches are scientifically robust. However, for the
experimental data, the values are obtained only for higher energies (>
1 MeV/u), and no significant new physics is expected, as in this
energy regime most of the well-known semi-empirical models and ab
initio calculations work satisfactorily well. This lack of
experimental evidence (at low energies) to verify the reliability of
the present model is indeed the major drawback of the present work.

Nevertheless, the core of the present work (i.e. a theoretical
approach capable to predict stopping power from low to high energies,
over several orders of magnitude in ion energy describing the Bragg
peak) is of broad enough interest for the readership of Physical
Review A. In this context, I recommend the manuscript to be accepted    
after minor revisions, especially to address more attention (and more
comments) to the aforementioned weakness of the manuscript.

I also request the authors to address attention to some comments to
further enhance the quality of the manuscript:

1. Can the authors discuss how possible impurities in the sample and –
especially - on both sides of the target would affect the final
stopping data? The authors must discuss it in the manuscript.

2. In the same context, the authors have to clarify whether potential
surface roughness and/or target non-uniformity might affect the final
results obtained in transmission method.

3. What is accuracy of the primary beam energy of the accelerator?
Moreover: when the target is placed and removed from the front of the
detector, does the detector itself need to unbiased and the chamber
opened? If so, might there be any offset (~keV) in the detection
system that would affect the energy distribution peaks, hence the
Gaussian position from the fits (Fig. 1)?

4. The experimental data above ~1.5 MeV is systematically below the
data from Ref. [7], hence SRIM semi-empirical approach, and also below
the authors’ theory output. At these energies (and higher), the
agreement between experiment and theory is expected to be better than
a few percent (that is why you used stopping power for 5.486 MeV alpha
on Hf to obtain the target thickness). Can the authors explain why
there is such systematic difference for the present measurements using
protons?

5. The authors claim an overall uncertainty of 5%, but it is not clear
how they obtained this value, or whether only statistical
contributions were taken into account. Nevertheless, uncertainties of
measurements have to be treated following the Guide to the Expression
of Uncertainty in Measurement, International Organization for
Standardization, Geneva Switzerland, 1995. Especially for stopping
power measurements in transmission approach, there are other sources
of uncertainties that might potentially affect the final results
systematically (see comments 1, 2 and 3), and they must be discussed
in more details.

6. Can the present theory in the manuscript be compared to other
models (e.g., CasP and DPASS), especially towards low energies? Aiming
thus a constructive discussion on the overall-agreement among
different models to predict SCS of “challenge” materials, as
transition metals (f-subshell)?

7. In the last sentence of Sec. IV, the authors point out for a need
of more experimental data below 100 keV H+ but, in my opinion, more
data are needed even at higher energies, e.g., nearby (and slightly
above) the Bragg peak... Please enhance discussions about it.

----------------------------------------------------------------------
Report of the Third Referee -- AP11756/Montanari
----------------------------------------------------------------------

Montarani and co-workers perform a joined experimental and theoretical
study of the stopping power (SP) of protons in Hafnium thin films. In
my opinion, the contribution of the present study is two-fold. Despite
its fundamental interest, experimental data on the SP of particles on
metals with large atomic number are scarce. Hence, the new
measurements reported in this work for 0.6-2.5 MeV protons in Hf,
which is characterized by occupied 4f and 5p shells, are valuable per
se. Moreover, the authors also provide new theoretical calculations of
the SP, in which they combine different models in order to cover an
ample range of proton energies. In particular, the theoretical study
mainly focuses on understanding the role of the excitation of 4f-5p
electrons in the SP. Overall, I consider this work is valuable to
extend our knowledge on the interaction of particles (protons in this
case) with heavy metals and, as such, it will be of interest for the
ion-solid interaction community.

In general the manuscript is well written, but there are some points
regarding the employed theoretical approximations that are a bit
confusing. Therefore, my recommendation is publishing in Physical
Review A once the authors clarify the following points:

- Regarding the theoretical model that describes the contribution to
the stopping power of the bound electrons, the authors emphasize the
importance of including relativistic effects in the description of the
binding energies and densities of the Hf 1s-4f states. However, it is
not clear if the calculations performed by the authors are done for
gas phase Hf atoms or for the solid Hf as in the experiments of Ref.
[38]. In case of being for the former, the slightly worse agreement
found for the 4f states as compared to the experimental values could
be related to this point.

- The results from the three different theoretical calculations shown
in Fig. 3 are not completely clear. The authors emphasize the
importance of considering both screening effects and relativistic
effects for the 4f and 5p electrons, but it is not clear if later in
the text the authors use indistinctly the terminology 'relativistic'
and 'screening' to refer to the same issue. For instance, are the ML
results shown in Fig.3 calculated with the relativistic binding
energies and densities, but with and without including screening
corrections, or do the two ML curves differ in whether the
relativistic corrections are or not included? It is neither clear what
description of the bound electrons is used in the curve labeled as
'SPCC(FEG)+SLPA(bound)'.

- If all the results of Fig. 3 were calculated for the relativistic
description, it would be also interesting to show the results obtained
with the non-relativistic description in order to demonstrate the
importance of this correction.

- Furthermore, it would also be meaningful to show the results from
the ML(FEG) calculation alone in order to identify better the
contribution of the bound electrons to the SP.

- When comparing the new and existing experimental data, clarify what
experimental technique was used in Ref.7 to measure the stopping
power.

- The data by Sirotonin et al. plotted in Fig.4 seem to suggest that
the stopping power maximum would be around 100-200 keV, while the
maximum predicted in the present work by the theoretical calculations
is located at slightly smaller energies. Please comment on this point
and discuss possible reasons for this discrepancy: What are the main
limitations in the theoretical model that can be causing this energy
shift?

- When reviewing on the experimental work performed on Hf oxides in
the introduction, note that there is a quite recent work by Roth and
co-workers published in Phys. Rev. Lett. 119, 163401 (2017), in which
the SP of protons is precisely measured in this target.

Finally, there are few misprints in the generated pdf file:

- Y-axis labels in Figs. 3 an 4 are unreadable

- Page 3,2nd column, 2nd full paragraph: "Too assess"

- Caption of Fig.2: "hollow circles" -> "open circles" or "unfilled
circles"
